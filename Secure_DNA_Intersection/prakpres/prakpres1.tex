\documentclass{beamer}
%
% Choose how your presentation looks.
%
% For more themes, color themes and font themes, see:
% http://deic.uab.es/~iblanes/beamer_gallery/index_by_theme.html
%
\mode<presentation>
{
   
  \usetheme{IMI} 
  %\usetheme{default}      % or try Darmstadt, Madrid, Warsaw, ...
  \usecolortheme{default} % or try albatross, beaver, crane, ...
  \usefonttheme{default}  % or try serif, structurebold, ...
  \setbeamertemplate{navigation symbols}{}
  \setbeamertemplate{caption}[numbered]

} 

\usepackage[english]{babel}
\usepackage[utf8]{inputenc}
\usepackage{enumitem}
\newlist{arrowlist}{itemize}{1}
\setlist[arrowlist]{label=$\Rightarrow$}
\usepackage{tikz}

\usepackage{mathtools}
\usepackage{amsmath}
\tikzset{
	every overlay node/.style={
		%draw=black,fill=white,rounded corners,
		anchor=north west,
	},
}
% Usage:
% \tikzoverlay at (-1cm,-5cm) {content};
% or
% \tikzoverlay[text width=5cm] at (-1cm,-5cm) {content};
\def\tikzoverlay{%
	\tikz[baseline,overlay]\node[every overlay node]
}%

\title[Privatly computing Set-Intersection]{Privatly computing Set-Intersection using Bloom Filters}
\author{Niklas Jobst}
\institute{TCS - Universität zu Lübeck}
\date{\today}

\begin{document}


% Uncomment these lines for an automatically generated outline.
%\begin{frame}{Outline}
%  \tableofcontents
%\end{frame}
\begin{frame}{Algorithmus basierend auf Elgamal}
	\footnotesize
	\vskip 0.3cm
	\tikzoverlay (n1) at (-0.5cm,3cm) {
		\begin{minipage}{0.45\textwidth}
			\begin{block}{\textbf{Client} }
				\begin{arrowlist}
					\item Erstellt Bloom Filter der Daten
					\item Verschlüsselt jede Stelle des Bloom Filters mittels ElGamal
					
				\end{arrowlist}
				\quad  \ \ $(R_i, S_i)=(g \textsuperscript{r\textsubscript{i}},pk\textsuperscript{r\textsubscript{i}}*g\textsuperscript{1-BF\textsubscript{1}[i]})$
			\end{block}
		\end{minipage}
	};
	
	\tikzoverlay (n2) at (6.7cm,3.5cm) {
		\begin{minipage}{0.45\textwidth}
			\begin{block}{\textbf{Server}}
				\begin{arrowlist}
					
					\item Erstellt Bloom Filter der Daten
					\item Selektiert jene Stellen Stellen in dem BF die den Eintrag null besitzen.
					\item Multipliziert an diesen Stellen die Werte des Ciphertextes vom Client auf 
					\item Rerandomisiert die entstandenen Ergebnisse
					\quad  \ \	$$ V = (g \textsuperscript{s} * \Pi_{i:BF_{2}[i] = 0} R_{i} )$$
					$$ W = (pk \textsuperscript{s} * \Pi_{i:BF_{2}[i] = 0} S_{i} )$$ \\
					
				\end{arrowlist}
			\end{block}
		\end{minipage}
	};
	
	\tikzoverlay (n3) at (-0.5cm,0cm) {
		\begin{minipage}{0.45\textwidth}
			\begin{block}{}
				\begin{arrowlist}
					\item Alice entschlüsselt mit sk Ciphertext von Bob
					\item Bestimmt Anzahl der Einträge  an denen beide Bloom Filter null sind
					\item Schätzt hieraus die Gesamtmenge an SNPs
				\end{arrowlist}
			\end{block}
		\end{minipage}
	};
	
	\begin{tikzpicture}[overlay]
	
	\draw[->,>=latex, line width=0.1cm] (4.8, 1)--(6.9, 1) node[above left] { $ \ [pk, (R_i, S_i)] \ \ $};
	\draw[->,>=latex, line width=0.1cm] (6.9, -1)--(4.8, -1) node[above right] {  $ \ \ (V,W) \ \ $};
	
	
	\end{tikzpicture}
	
	
\end{frame}


\begin{frame}{Abschätzung der Elemente in einem Bloomfilter}
	$$ |X| = \frac{ln( \frac{z}{m})}{k* ln(1- \frac{1}{m})}$$
	
	\begin{arrowlist}
		\item Basiert auf Abschätzung der Anzahl an Nullen in einem Bloomfilter in einem Bloomfilter:
		\item 	$ z = m*(1 -\frac{1}{m})\textsuperscript{k* X}$ 
	\end{arrowlist}
	
\end{frame}

\begin{frame}
		\begin{arrowlist}
			\item Wahrscheinlichkeit, dass ein Bloomfilterbit Null ist $ z' = (1 -\frac{1}{m})\textsuperscript{k* X} $
			\item Da biomialverteilt ist der Erwartungswert:
			$ z = m*(1 -\frac{1}{m})\textsuperscript{k* X}$ 
			\item 
		\end{arrowlist}
\end{frame}

\section{Algorithmus}


\begin{frame}{Paillier - Verfahren}

	\textbf{Schlüsselerzeugung:}\\
	
	\begin{arrowlist}
	\item Client wählt zwei Primzahlen p,q , mit $ ggt(pq, (p-1)(q-1))= 1$  
	\item Der Generator g so gewählt, sodass $ g \in (\mathbb{Z} n^{2} \mathbb{Z}) $ und$ n $ die Ordnung von g teilt.	
	\item Secrect key: $ \lambda = kgV(p-1, q-1) $
	\item Public Key: $(n,g)$	
	\end{arrowlist}
	
	

\end{frame}

\begin{frame}
		\textbf{Verschlüsselung:}
		\begin{arrowlist}
			\item Client wählt Zufallszahl $ r $ wobei $ 0 \leq r \leq n $
			\item Ciphertext $ c = g^{m}*r^{n} \mod\ n^{2} $
		\end{arrowlist}
		

		\textbf{Entschlüsselung:}
		\begin{arrowlist}
			\item Benötigt zunächst $ L(u)= \frac{(u-1)}{n} $
			\item Plaintext $ m = \frac{L(c^{\lambda} \ mod \ n^{2}) }{L(g^{\lambda} \ mod \ n^{2})} \ mod \ n $
		\end{arrowlist}


		\textbf{Homomorphie:}
		Paillier ist homomorph gegenüber der Addition.
		
		$$ E(m \textsubscript{1} + m \textsubscript{2}) = (E(m \textsubscript{1}) + E(m \textsubscript{2}))$$
\end{frame}

\begin{frame}{Algorithmus basierend auf Paillier}
	\footnotesize
	\vskip 0.3cm
	\tikzoverlay (n1) at (-0.5cm,3cm) {
		\begin{minipage}{0.45\textwidth}
			\begin{block}{\textbf{Client} }
				\begin{arrowlist}
						\item Erstellt Bloomfilter der Daten und invertiert jede Stelle des Bloomfilters.
						\item Verschlüsselt jede Stelle des Bloomfilters mittels Paillier
						
					\end{arrowlist}
					\quad  \ \ $ \ c_{1-m} =(g^{m}*r^{n}) \ mod \ n^{2} $
			\end{block}
		\end{minipage}
	};
	
	\tikzoverlay (n2) at (6.7cm,3.5cm) {
		\begin{minipage}{0.45\textwidth}
			\begin{block}{\textbf{Server}}
				\begin{arrowlist}
					
						\item Erstellt für jedes Element des Datensatzes einen Bloomfilter seiner Daten
						\item Selektiert in jedem Blommfilter jene Stellen die den Eintrag Eins besitzen.
						\item Addiert an diesen Stellen die Werte des Ciphertextes des Clients auf 
						\item Rerandomisiert die entstandenen Ergebnisse mit verschlüsselter Null
						\quad  \ \	$$ Rerand \ e_{j} = (ej \ *  \ enc_{pailier}(0) )$$

					
				\end{arrowlist}
			\end{block}
		\end{minipage}
	};
	
	\tikzoverlay (n3) at (-0.5cm,0cm) {
		\begin{minipage}{0.45\textwidth}
			\begin{block}{}
				\begin{arrowlist}
					\item Client entschlüsselt mit sk Ciphertexte von Server
					\item Anzahl der entschlüsselten Nullen entspricht der Anzahl der sich überschneidenden Elemente
				\end{arrowlist}
			\end{block}
		\end{minipage}
	};
	
	\begin{tikzpicture}[overlay]
	
	\draw[->,>=latex, line width=0.1cm] (4.8, 1)--(6.9, 1) node[above left] { $  [pk, c] \ \ \ \ $};
	\draw[->,>=latex, line width=0.1cm] (6.9, -1)--(4.8, -1) node[above right] {  $ \ \ rerand(e_{1-j}) \ \ $};
	
	
	\end{tikzpicture}
	
	
\end{frame}

\begin{frame}{Ergebnisse - Elgamal }
	
	\begin{arrowlist}
		\item Dauer für Vergleich des gesamten Exomes bei wenigen Minuten.
		\item Laufzeit Unabhängig davon wie stark die Überschneidung zwischen zwischen den Datensätzen ist. 
	\end{arrowlist}

 \begin{table}[h]
	\begin{tabular}{c|c|c|c|c}
		Überschneidung&14000&7500&5000&2000\\
		\hline
		Runtime (sec)& 221&247&211&222\\
		Abw. zur Überschn.& 0.01\%& 3.3\%&8.8\%&36.8\%\\
			 		
	\end{tabular}
	\caption{Hashfunktionen : 14, Anzahl Bloomfilter Bits:3029660, Größe der Datensätze: 15000 SNPs }
	\label{tab:meinetabelle1}
	
	
\end{table}


\end{frame}	
\begin{frame}
\begin{table}[h]
			 	
	\begin{tabular}{c|c|c|c|c}
	    Array& 1442696&1009887&577079&144270\\
	    \hline
		Runtime (sec)& 108&83&47&11\\
		Abweichung&4\%&6\%&13\%&51\%\\

			 		
	\end{tabular}
	\caption{Datensatz 1000 SNPs, Überschneidung 100, Hashfunktionen: 10 }
	\label{tab:meinetabelle2}
\end{table}

\begin{arrowlist}
	\item Die Laufzeit ist linear abhängig zur Anzahl der Bloomfilterbits
	\item Die Stärke der Abweichung ist ebenfalls linear abhängig zur Anzahl der Bloomfilter Bits
\end{arrowlist}
\end{frame}			 
	
\begin{frame}
	\begin{table}[h]
		
		\begin{tabular}{c|c|c|c|c|c}
			Hashf.&1&4&7&10&14\\
			\hline
			Runtime (sec)&7&27&44&62&104\\
			Abweichung&11\%&13\%&10\%&9\%&9\%\\
			
			
		\end{tabular}
		\caption{Datensatz 1000 SNPs, Überschneidung 100, Array: 504944 }
		\label{tab:meinetabelle4}
	\end{table}
	
	\begin{arrowlist}
		\item Anzahl der Hashfunktionen hat deutlich weniger Einfluss, jedoch kommt es bei hoher Anzahl zu vermehrt Falsch positiven Ergebnissen.
	\end{arrowlist}
\end{frame}


\begin{frame}{Ergebnisse-Paillier}
	\begin{table}[h]
		
		\begin{tabular}{c|c|c|c|c|c}
			Array&14139&12119&10099&8080\\
			\hline
			Runtime (sec)&219&194&183&163\\
			Abweichung&1\%&4\%&6\%&24\%\\
			
			
		\end{tabular}
		\caption{Hashf.7, Überschneidung 100, SNPs 1000 }
		\label{tab:meinetabelle5}
	\end{table}
	
	\begin{arrowlist}
		\item Paillier deutlich langsamer als Elgamal
		\item Benötigt deutlich kleinere Bloomfilter für selbe Genauigkeit, jedoch ist die Bitweise Verschlüsselung sehr langsam
	\end{arrowlist}
\end{frame}
\begin{frame}
	\begin{table}[h]
		
		\begin{tabular}{c|c|c|c|c|c}
			Array&141385&100989&75742\\
			\hline
			Runtime (sec)&2420&2318&2007\\
			Abweichung&1\%&4\%& 13\%\\
			
			
		\end{tabular}
		\caption{Hashf.7, Überschneidung 7500, SNPs 15000 }
		\label{tab:meinetabelle6}
	\end{table}
	
	\begin{arrowlist}
		\item Zum Vergleich von gesamten Exomen ca. 40 min
		\item Aufgrund der kleineren Bloomfiltergröße kommt es jedoch nicht so schnell zum Überlauf des Arbeitsspeichers.

		
	\end{arrowlist}

\end{frame}

\begin{frame}{Vergleich}
	\begin{table}[h]
		
		\begin{tabular}{c|c|c|c|c|c|c}
			Abweichung&0.1\%&0.6\%&2\%&3\%&4\%&6\%\\
			\hline
			Runtime elgamal&467&150&17&15&11&6\\
			Runtime paillier&510&340&150&150&135&120\\
			
			
		\end{tabular}
		\caption{Hashf.7, Überschneidung 100, SNPs 1000 }
		\label{tab:meinetabelle7}
	\end{table}

	
\end{frame}
\end{document}

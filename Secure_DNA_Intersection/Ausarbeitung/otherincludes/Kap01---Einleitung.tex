\chapter{Introduction}
\label{sec:Chapter1}


\section{ Divry-Van Bogaert syndrome}
\label{sec:Sec1.1}

\subsection{State of the art}
\label{sec:Sec1.1.1}

\piccaption{Ludo van Bogaert \cite{ludo}} 
\parpic[r]{\includegraphics [width=4.5cm, height=6.5cm]{Ludo_van_bogaert.jpg}}


The Divry-van Bogaert syndrome (DBS)  was first described in 1946 by the two eponymous belgian doctors Paul Divry and Ludo van Bogaert. They studied and described two siblings, with a clinical picture marked by livedo racemosa, cerebral
angiomatosis, dementia and epilepsy \cite{sds, cauf, dvf, dvbp}.

DBS is an extremely rare disease.
Since its discovery, very few cases have been reported. An article from the year 2001 determined a total of only 13 patients worlwide\cite{cauf, dvf}. Another argument for its rarity is that PubMed currently has only eight entries mentioning the syndrome\cite{pubmed1}. 

With only a few cases, there hasn't been much research done on the syndrome in the past. Therefore, its pathogenic cause and its exact clinical classification are still uncertain. 
Nonetheless there have been a couple of new scientific findings on the topic since its discovery:\\
First papers about DBS discussed whether the syndrome is sporadic or congenital\cite{dvf,cauf}, later publications identified DBS as a recessive hereditary disease \cite{dvbp}. Up to the millennium it was assumed that the disease only affects males, but recently a female case was reported. However, the syndrome still seems to affect predominantly men. Further on, it is also discussed whether DBS overlaps with other rare diseases, mainly with the sneddon syndrome\cite{sds}. Both diseases share similar symptoms. It has even been suggested that the Divery van Bogeart syndrome belongs to the sneddon syndrome. 
However, the sneddons syndrome is mainly affecting females and its symptoms occur at another age \cite{sds}. 

\subsection{Symptoms}
\label{sec:Sec1.1.2}

The Divry-Van Bogaert syndrome is considered to be part of the neurocutane syndromes with predominant vascular symptoms.
Neurocutane syndromes are diseases defined by manifestions on skin and in the nervous system. \cite{doc,neurocutane_all,neurocutane}
In the case of DBS the manifestions in the skin is livedo racemose and the main symptom of the nervous system cerebral angiomatosis\cite{sds, cauf, dvf, dvbp}.

Cerebral angiomatosis describes a condition, which is characterised by multiple angiomas of the cerebral arteries.
These angiomas consist of blood vessel branches, forming diffuse structures in shape of little knots \cite{cauf}. 
In DBS, angiomatosis occurs mainly in the distal regions of anterior and middle cerebral arteries or distal, fine angiomatous circulation from tiny collateral vessels in distal arterial occlusions \cite{cauf,corr}. 



\chapter*{Abstract}


Ziel dieses Praktikums war es zu erörtern, wie zwei Parteien die Ähnlichkeit ihrer DNA  berechnen können, ohne, dass dabei eine der Parteien Informationen über den genetischen Code der jeweils anderen erlangt.

Die Grundlagen für diese Berechnungen basieren auf bereits existierenden Methoden, mit welchen der Schnitt zweier Mengen unter Sicherung der Privatsphäre berechnet werden kann.

Im Zuge dieses Praktikums habe ich drei dieser Methoden mit Bezug zum gegebenen Anwendungsfall implementiert und deren Effizienz miteinander vergleichen:


\begin{itemize}
	\item R.Egert et al. : Privately Computing Set-Union and Set-Intersectio Cardinality via Bloom Filters, LNCS volume 9144, 2015
	\item A.Davidson et al. : An Efficient Toolkit for Computing Private Set Operations, LNCS volume 10343, 2017
	\item S. K.Debnath et al. : Secure and Efficient Private Set Intersection Cardinality Using Bloom Filter, LNCS volume 9290, 2015
\end{itemize}

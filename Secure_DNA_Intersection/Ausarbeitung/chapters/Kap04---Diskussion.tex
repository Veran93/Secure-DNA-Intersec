\chapter{Diskussion}

Um die Genauigkeit und Effizienz der Algorithmen zu ermitteln und zu vergleichen, wurden diese auf drei unterschiedlich große Test Datensätze angewendet. 
Der größte dieser Datensätze entspricht dabei 15000 SNPs, was ungefähr der Anzahl an SNPs in einem menschlichen Exom entspricht.
Der mittlere Datensatz enthält 1000 SNPs und der kleinste besitzt eine Größe von 100 SNPs. 

Bei den Tests wurde wurden verschiedenen Parameter verschoben um deren Einfluss auf die Ergebnisse zu bestimmen.
Hierunter befinden sich die Größe des gewählten Boomfilters, der Anzahl der Hash-Funktionen sowie die Überschneidung der der gewählten Datensätze.

\section{Elgamal basierter Algorithmus}

Die Tabelle zeigt wie sich die Laufzeit und Genauigkeit bei unterschiedlichen großem Bloomfilter Array. 
Zu beiden Parametern ist ein linearer Zusammenhang zu erkennen.

Anders sieht es bei unterschiedlicher Anzahl der Hashfunktionen aus. 
Wie in der Tabelle zu sehen ist, deren Anzahl kaum Einfluss auf die Genauigkeit des Algorithmus.

Bei Anwendung auf den Exom Großen Datensatz benötigte der Algorithmus durchgehend eine Laufzeit von ca. 3,7 Minuten.
Da dies die die wohl größten zu erwartenden Datensätze sind, lassen sich alle nötigen Vergleiche in annehmbarer Zeit berechnen. 


\section{Paillier basierter Algorithmus}


\section{Vergleich der Algorithmen}

Paillier deutlich langsamer als Elgamal. Benötigt kleinere Bloomfilter bei selber Genauigkeit, jedoch ist die Bitweise Verschlüsselung sehr langsam.

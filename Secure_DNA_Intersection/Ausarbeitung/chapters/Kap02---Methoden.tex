\chapter{Methoden}
\label{sec:Chapter2}
\section{Bloom Filter}
\label{sec:Sec1.2}

Alle diese Methoden basieren auf sogenannten Bloomfiltern.
Hierbei handelt es sich um eine  Technik um festzustellen, ob bestimmte Daten in einem Datensatz vorhanden sind oder nicht.
Sie bestehen aus einem mit Nullen vorinitialisiertem m Bit langen Array und k Hashfunktionen, welche auf die Positionen des Arrays abbilden.

Zur Initialisierung werden auf jedes Element des Datensatzes alle k Hashfunktionen angewendet.
Die zur Ausgabe der Hashfunktionen korrespondierenden Bits im Array werden darauf hin auf Eins gesetzt.

Soll für ein Datenelement geprüft werden, ob dieses Teil des Datensatzes ist, werden  alle Hashfunktionen auf dieses angewendet.\\
Nur wenn alle Positionen im Array an den korrespondierenden Punkten der Ausgabe dem Wert Eins entsprechen wird angenommen das sich das Element im Datensatz befindet.

Diese Überprüfung ist nicht resistent gegenüber  

\section{Kryptosysteme}
\label{sec:Sec1.3}

\subsection{Homomorphie}
Homomorphie bezeichnet eine Eigenschaft von Kryptosystemen. 
Ein Kryptosystem ist genau dann homomorph gegenüber einer mathematischen Operation, wenn Berechnungen im Ciphertext mit dieser Operation denen im KLartext entsprechen.

\subsection{Elgamal}
\label{sec:Sec1.3.1}
\begin{enumerate}
	\item Bei Elgamal handelt es sich um ein im Jahr 1985 vom Kryptologen Taher Elgamal entwickeltes Public-Key-Verschlüsselungsverfahren. Elgamal ist eine Erweiterung des Diffie-Hellmann Schlüsselaustausches.
\end{enumerate}

\subsubsection{Schlüsselerzeugung}
Zunächst wählt der Client eine endliche zyklische Gruppe Z der Ordnung q mit einem Generator g.
\begin{itemize}	
	\item Secrect key: Der Client wählt eine zufällige Zahl $a<q$ mit dem $GGT(a,q) = 1$. Dies ist der Secret key
	\item Public Key: Der public key ist dann $P = g\textsuperscript{a} $	
\end{itemize}
\subsubsection{Verschlüsselung}
Sei $m \in Z\textsubscript{q}$ die zu versendende Nachricht. Dann wählt der Server eine zufällige Zahl $r<q$ mit dem $GGT(r,q) = 1$. Nun berechnet sich $c\textsubscript{1} = g\textsuperscript{r}$ sowie $c\textsubscript{2} =P \textsuperscript{r} *m$. Der Ciphertext besteht so aus $C = (c\textsubscript{1},c\textsubscript{2})$.
\subsubsection{Entschlüsselung}
Zur Entschlüsselung wird $\Sigma = c_{1}^{-q} * c_{2} brechnet.$
\subsubsection{Homomorphie }
Elgamal ist homomorph gegenüber der Multiplikation
$$ E(m \textsubscript{1} * m \textsubscript{2}) = (E(m \textsubscript{1}) * E(m \textsubscript{2}))$$
\subsubsection{Sicherheit}
\subsubsection{Zuverlässigkeit}


\subsection{Pailier}
\label{sec:Sec1.3.2}

Das Schlüsselpaar wird folgendermaßen generiert:


\subsection{Goldwasser-micali}
\label{sec:Sec1.3.3}

\section{Implementierte Algorithmen}

\subsection{Algorithmus 1 - Elgamal}

Der hier beschriebene Algorithmus wurde zunächst im ... veröffentlicht. Es wurden Algorithmen für unterschiedliche Konstellationen postuliert. An dieser Stelle habe ich den zwei Parteien Fall genutzt.

Der Client erstellt zu Beginn einen Boomfilter seiner Daten. Dabei wird jedes Datenelement einzeln 
Zur Verschlüsselung wählt der Client zunächst public und secret key nach elgamal.
Daraufhin wird jedes Bit des Bloomfilter Arrays einzeln verschlüsselt.
Hierzu werden die zu sendende Nachrichten so gewählt das $m = g^{1-BF[i]}$
Dies führt dazu, dass m an Stellen an welchen der Bloomfiltern einen 1 besitzt einer 1 entspricht und g wenn er dort eine 0 besitzt.
Dann berechnet er seine Mit diesen

$$(R_i, S_i) = (g \textsuperscript{r\textsubscript{i}} , pk\textsuperscript{r\textsubscript{i}} * g\textsuperscript{1-BF\textsubscript{1}[i]})$$


\[
S\textsubscript{i} = pk^{r_{i}} \ * \ \left\{
\begin{array}{ll}
g^{0} = 1 \ bei \  BF\textsubscript{1}[i]=1\\
g^{1} = g \ bei \ BF\textsubscript{1}[i]=0\\
\end{array}
\right.
\]

Aufmultiplikation  von $R_i$ bzw $S_i$ an jenen Stellen, an welchen $BF_2 = 0$ ist.
Rerandomisierung mit $g^s$ bzw. $pk^s$ 
Diese Berechnungen sind ohne Datenverlust aufgrund der Homomorphie Eigenschaft von Elgamal möglich
$$ V = (g \textsuperscript{s} * \Pi_{i:BF_{2}[i] = 0} R_{i} )$$
$$ W = (pk \textsuperscript{s} * \Pi_{i:BF_{2}[i] = 0} S_{i} )$$ \\


$R_{i}, \  S_{i}$ aus vorherigem Schritt in V,W einsetzen.
	


$$ V = (g^{s + r_{i_{1}} + r_{i_{2}} + \ ...\ +r_{i_{k}}})$$

\[
W =\left\{
\begin{array}{ll}
pk^{s + r_{i_{1}} + r_{i_{2}} + \ ...\ +r_{i_{l}}}*1 \ falls \ BF_{1} = 1,BF_{2} = 0 \\
pk^{s + r_{i_{1}} + r_{i_{2}} + \ ...\ +r_{i_{m}}}*g^{x} \ falls \ BF_{1} = BF_{2} = 0\\
\end{array}
\right.
\]

$$ W= (pk^{s + r_{i_{1}} + r_{i_{2}} + \ ...\ +r_{i_{k}}}* g^x)$$


\subsection{Algorithmus 2 - Paillier}
\label{sec:Sec2.3}


\subsection{Algorithmus 3 - Goldwasser-Micali}
\label{sec:Sec2.4}
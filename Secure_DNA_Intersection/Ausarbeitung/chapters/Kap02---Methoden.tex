\chapter{Methoden}
\label{sec:Chapter2}
\section{Bloom Filter}
\label{sec:Sec1.2}

Alle diese Methoden basieren auf sogenannten Bloomfiltern.
Hierbei handelt es sich um eine  Technik um festzustellen, ob bestimmte Daten in einem Datensatz vorhanden sind oder nicht.
Sie bestehen aus einem mit Nullen vorinitialisiertem m Bit langen Array und k Hashfunktionen, welche auf die Positionen des Arrays abbilden.

Zur Initialisierung werden auf jedes Element des Datensatzes alle k Hashfunktionen angewendet.
Die zur Ausgabe der Hashfunktionen korrespondierenden Bits im Array werden darauf hin auf Eins gesetzt.

Soll für ein Datenelement geprüft werden, ob dieses Teil des Datensatzes ist, werden  alle Hashfunktionen auf dieses angewendet.\\
Nur wenn alle Positionen im Array an den korrespondierenden Punkten der Ausgabe dem Wert Eins entsprechen wird angenommen das sich das Element im Datensatz befindet.

Diese Überprüfung ist nicht resistent gegenüber  

\section{Kryptosysteme}
\label{sec:Sec1.3}

\subsection{Homomorphie}
Homomorphie bezeichnet eine Eigenschaft von Kryptosystemen. 
Ein Kryptosystem ist genau dann homomorph gegenüber einer mathematischen Operation, wenn Berechnungen im Ciphertext mit dieser Operation denen im KLartext entsprechen.

\subsection{Elgamal}
\label{sec:Sec1.3.1}
Bei Elgamal handelt es sich um ein im Jahr 1985 vom Kryptologen Taher Elgamal entwickeltes Public-Key-Verschlüsselungsverfahren. Elgamal ist eine Erweiterung des Diffie-Hellmann Schlüsselaustausches.

Elgamal ist homomorph gegenüber der Multiplikation
$$ E(m \textsubscript{1} * m \textsubscript{2}) = (E(m \textsubscript{1}) * E(m \textsubscript{2}))$$

\subsection{Pailier}
\label{sec:Sec1.3.2}



\subsection{Goldwasser-micali}
\label{sec:Sec1.3.3}

\section{Implementierte Algorithmen}

\subsection{Algorithmus 1 - Elgamal}
\label{sec:Sec2.2}

\subsection{Algorithmus 2 - Paillier}
\label{sec:Sec2.3}


\subsection{Algorithmus 3 - Goldwasser-Micali}
\label{sec:Sec2.4}
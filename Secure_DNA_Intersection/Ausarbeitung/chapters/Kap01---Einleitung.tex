\chapter{Einleitung}
\label{sec:Chapter1}

\section{Next generation sequencing}
In den ersten Jahren des 21.Jahrhunderts wurden mehrere Sequenzierungstechnicken entwickelt, welche unter dem Begriff next generation sequencing (NGS) zusammengefasst werden.
Sie alle vereint die Fähigkeit, DNA in großem Maßstab parallel sequenzieren zu können. 
Dies hatte zur Folge, dass NGS Sequenzierungen im Vergleich zur klassischen Sanger Sequenzierungen in den darauf folgenden Jahren deutlich schneller und günstiger wurden.
Kostete die Sequenzierung eines Genoms 2000 noch über 10 Millionen US Dollar, sind die Preise bis zum heutigen Tag auf unter 1000 gefallen.
  
\section{Genetisch marker}
In diesem Projekt wird die DNA der beiden Parteien als Mengen betrachtet.
Aufgrund der Tatsache, dass der Großteil der DNA bei allen Menschen identisch ist, nutzte ich genetische Marker, welche die DNA unterscheiden.
Der Schnitt dieser beiden Marker dient  dann als Maß der Ähnlichkeit  der jeweiligen DNAs.
\subsection{Genetische Marker}

Unter genetischen Markern werden bestimmte klar definierte Sequenzen und Positionen im genetischen Code können dazu genutzt werden Personen zu identifizieren.

\subsubsection{SNPs}


\subsubsection{INDELs}

\subsection{Personalisierte Medizin}
In der personalisierte Medizin werden individuelle Eigenschaften von Personen berücksichtigt die  



\section{Anwendung}

\subsection{Personalisierte Medizin}
In der personalisierte Medizin werden individuelle Eigenschaften von Personen berücksichtigt, insbesondere genetische 
In der Personalisierten Medizin sind  Therapien bestimmte genetische Profile  gekoppelt.
Um festzustellen, ob eine Therapie für einen Patienten zulässig ist, muss daher zunächst sein genetischer Code mit dem für diese Therapie notwendigem verglichen werden.
Derzeit werden diese Vergleiche ohne die entsprechenden Datensicherheits-Vorkehrungen vorgenommen.
Ziel dieses Praktikums war es durch Anwendung der genannten Methoden die Sicherung der Privatsphäre bei der Durchführung eines solchen Vergleichs zu erhöhen.


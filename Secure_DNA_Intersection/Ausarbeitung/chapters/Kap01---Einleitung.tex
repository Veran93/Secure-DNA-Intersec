\chapter{Einleitung}
\label{sec:Chapter1}


\section{ Abstract}
\label{sec:Sec1.1}

Ziel dieses Praktikums war es die Frage zu erörtern, wie zwei Partein die Ähnlichkeit ihrer DNA  berechnen können, ohne, dass dabei eine der Parteien Informationen über den genetischen Code der jeweils anderen erlangt.

Die Grundlagen für diese Berechnungen basieren auf bereits existierenden Methoden, mit welchen der Schnitt zweier Mengen unter Sicherung der Privatsphäre berechnet werden kann.

Im Zuge dieses Praktikums werden wir drei dieser Methoden mit Bezug zum gegebenen Anwendungsfall implementieren und deren Effizienz miteinander vergleichen:


\begin{itemize}
\item R.Egert et al. : Privately Computing Set-Union and Set-Intersectio Cardinality via Bloom Filters, LNCS volume 9144, 2015
\item A.Davidson et al. : An Efficient Toolkit for Computing Private Set Operations, LNCS volume 10343, 2017
\item S. K.Debnath et al. : Secure and Efficient Private Set Intersection Cardinality Using Bloom Filter, LNCS volume 9290, 2015
\end{itemize}

\section{Genetische Marker}
Bestimmte klar definierte Sequenzen und Positionen im genetischen Code können dazu genutzt werden Personen zu identifizieren.
Der genetische Code ist bei allen Menschen zu ca.99\% gleich. ogenannte 



\subsection{SNPs}


\subsection{INDELs}

\section{Personalisierte Medizin}
In der personalisierte Medizin werden individuelle Eigenschaften von Personen berücksichtigt die  

\section{Anwendung}
In der Personalisierten Medizin sind  Therapien bestimmte genetische Profile  gekoppelt.
Um festzustellen, ob eine Therapie für einen Patienten zulässig ist, muss daher zunächst sein genetischer Code mit dem für diese Therapie notwendigem verglichen werden.
Derzeit werden diese Vergleiche ohne die entsprechenden Datensicherheits-Vorkehrungen vorgenommen.
Ziel dieses Praktikums war es durch Anwendung der genannten Methoden die Sicherung der Privatsphäre bei der Durchführung eines solchen Vergleichs zu erhöhen.



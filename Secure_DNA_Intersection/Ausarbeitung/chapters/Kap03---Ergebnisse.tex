\chapter{Ergebnisse}

Zum Test der Algorithmen habe ich verschiedene Mengen an Test SNPs erstellt, die sich zu unterschiedlichen Graden überschneiden.


Beide Algorithmen konnten selbst große Datensätze schnell vergleichen.

\section{Algo elga}

 \begin{table}[h]
 	\begin{tabular}{c|c|c|c|c}
 		Überschneidung&14000&7500&5000&2000\\
 		\hline
 		Runtime (sec)& 221&247&211&222\\
 		Abw. zur Überschn.& 0.01\%& 3.3\%&8.8\%&36.8\%\\
 		
 	\end{tabular}
 	\caption{Hashfunktionen : 14, Anzahl Bloomfilter Bits:3029660, Größe der Datensätze: 15000 SNPs }
 	\label{tab:meinetabelle1}
 	
 	
 \end{table}
 
 \begin{table}[h]
 	
 	\begin{tabular}{c|c|c|c|c}
 		Array& 1442696&1009887&577079&144270\\
 		\hline
 		Runtime (sec)& 108&83&47&11\\
 		Abweichung&4\%&6\%&13\%&51\%\\
 		
 		
 	\end{tabular}
 	\caption{Datensatz 1000 SNPs, Überschneidung 100, Hashfunktionen: 10 }
 	\label{tab:meinetabelle2}
 \end{table}
\section{Algo 1}

	\begin{table}[h]
		
		\begin{tabular}{c|c|c|c|c|c}
			Array&14139&12119&10099&8080\\
			\hline
			Runtime (sec)&219&194&183&163\\
			Abweichung&1\%&4\%&6\%&24\%\\
			
			
		\end{tabular}
		\caption{Hashf.7, Überschneidung 100, SNPs 1000 }
		\label{tab:meinetabelle5}
	\end{table}
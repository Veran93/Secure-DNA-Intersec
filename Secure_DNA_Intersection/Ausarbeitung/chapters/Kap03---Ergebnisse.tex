\chapter{Ergebnisse}

Zum Test der Algorithmen habe ich verschiedene Mengen an Test SNPs aus der dbsnp gewählt, welche sich zu unterschiedlichen Graden überschneiden.

Insgesamt wurden zwei Datensätze benutzt. Einer mit tausend SNPs und einer mit 14000 SNPs, welcher in der Größe in etwa der Anzahl an SNPs im menschlichen Exom entspricht.

Die Überschneidungen der Datensätze wurde schrittweise zwischen 10 und 93 Prozent gewählt.
Des weiteren wurde auch der Einfluss Menge an Hashfunktionen sowie der Blooomfilter Größe geprüft.



\section{Algo elgamal}

 \begin{table}[h]
 	\begin{tabular}{c|c|c|c|c}
 		Überschneidung&14000 SNPs&7500 SNPs&5000 SNPs&2000SNPs\\
 		\hline
 		Runtime (sec)& 221&247&211&222\\
 		Abw. zur Überschn.& 0.01\%& 3.3\%&8.8\%&36.8\%\\
 		
 	\end{tabular}
 	\caption{Hashfunktionen : 14, Anzahl Bloomfilter Bits:3029660, Größe der Datensätze: 15000 SNPs }
 	\label{tab:meinetabelle1}
 	
 	
 \end{table}
 
 \begin{table}[h]
 	
 	
 	\begin{tabular}{c|c|c|c|c}
 		Array& 1442696&1009887&577079&144270\\
 		\hline
 		Runtime (sec)& 108&83&47&11\\
 		Abweichung&4\%&6\%&13\%&51\%\\
 		
 		
 	\end{tabular}
 	\caption{Datensatz 1000 SNPs, Überschneidung 100, Hashfunktionen: 10 }
 	\label{tab:meinetabelle2}
 \end{table}
 
\section{Algo 1}

	\begin{table}[h]
		
		\begin{tabular}{c|c|c|c|c|c}
			Array&14139&12119&10099&8080\\
			\hline
			Runtime (sec)&219&194&183&163\\
			Abweichung&1\%&4\%&6\%&24\%\\
			
			
		\end{tabular}
		\caption{Hashf.7, Überschneidung 100, SNPs 1000 }
		\label{tab:meinetabelle5}
	\end{table}